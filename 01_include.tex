\usepackage{tikz} % Drawing
\usepackage{pgfplots} % Axes
\pgfplotsset{compat=1.16}
\usepackage{relsize} % Font size
\usepackage{scalefnt}
\usepackage{ifthen}
\usetikzlibrary{
    calc, patterns, angles, quotes, backgrounds,
    animations,
    decorations.pathmorphing, decorations.pathreplacing, decorations.markings,
    arrows.meta, shapes.geometric,
}

% Color to white
\makeatletter
\newcommand{\globalcolor}[1]{%
  \color{#1}\global\let\default@color\current@color
}
\makeatother
\AtBeginDocument{\globalcolor{white}}


% Xparse: Optional argument
\usepackage{xparse}
\makeatletter
\define@key{xarg}{x}{\def\xx{#1}}
\define@key{xarg}{y}{\def\xy{#1}}
\define@key{xarg}{shift}{\def\xshift{#1}}
\define@key{xarg}{scale}{\def\xscale{#1}}
\define@key{xarg}{rotate}{\def\xrotate{#1}}
\define@key{xarg}{color}{\def\xcolor{#1}}
\define@key{xarg}{animate}{\def\xanimate{#1}}
\define@key{xarg}{xshift}{\def\xyshift{#1}}
\define@key{xarg}{yshift}{\def\xxshift{#1}}
\makeatother

\definecolor{orange}{HTML}{FF7F00}

\definecolor{gb-bright-red}{HTML}{fb4934}
\definecolor{gb-bright-green}{HTML}{b8bb26}
\definecolor{gb-bright-yellow}{HTML}{fabd2f}
\definecolor{gb-bright-blue}{HTML}{83a598}
\definecolor{gb-bright-purple}{HTML}{d3869b}
\definecolor{gb-bright-aqua}{HTML}{8ec07c}
\definecolor{gb-bright-orange}{HTML}{fe8019}

\definecolor{gb-neutral-red}{HTML}{cc241d}
\definecolor{gb-neutral-green}{HTML}{98971a}
\definecolor{gb-neutral-yellow}{HTML}{d79921}
\definecolor{gb-neutral-blue}{HTML}{458588}
\definecolor{gb-neutral-purple}{HTML}{b16286}
\definecolor{gb-neutral-aqua}{HTML}{689d6a}
\definecolor{gb-neutral-orange}{HTML}{d65d0e}

\definecolor{gb-faded-red}{HTML}{9d0006}
\definecolor{gb-faded-green}{HTML}{79740e}
\definecolor{gb-faded-yellow}{HTML}{b57614}
\definecolor{gb-faded-blue}{HTML}{076678}
\definecolor{gb-faded-purple}{HTML}{8f3f71}
\definecolor{gb-faded-aqua}{HTML}{427b58}
\definecolor{gb-faded-orange}{HTML}{af3a03}

\NewDocumentCommand{\tincoord}{O{}}{%
    % Origin, Float, sight, source
    \coordinate (X) at (10, 0);
    \coordinate (Y) at (-10, 0);
    \coordinate (Z) at (5, 0);
    \coordinate (X2) at (11, 0);
    \coordinate (cross) at (-6, 8);
    \coordinate (S) at (20, 20);
}




%\DeclareExpandableDocumentCommand{\fishingfloat}{O{}}{%
\NewDocumentCommand{\fishingfloat}{O{}}{%
    % Fishing float
    % Size: 2, 6
    \setkeys{xarg}{x={0}, y={0}, scale={1}, rotate={0}, #1}%
    \begin{scope}[shift={(\xx, \xy)}, scale=\xscale, rotate=\xrotate]
    % Triangle
    \draw (0, -3) -- (1, 1);
    \draw (0, -3) -- (-1, 1);
    % 2 lines
    \draw (-1, 1) -- (1, 1);
    \draw (-0.9375, 0.75) -- (0.9375, 0.75);
    % Top bottom
    \draw (0, 2) -- (0, 2.5);
    \draw (0, -3) -- (0, -3.5);
    % Arc: start angle, end angle, radius
    \draw (1, 1) arc (0:180:1);
    \end{scope}
}

\NewDocumentCommand{\stone}{O{}}{%
    % Stone (to thow in the water
    % Size: 0.2, 0.2
    \setkeys{xarg}{x={0}, y={0}, scale={1}, rotate={0},#1}%
    \begin{scope}[shift={(\xx, \xy)}, scale=\xscale, rotate=\xrotate, transform shape]
    %\begin{scope}[shift={(#1)},rotate=#2,scale=#3,transform shape]
    % \draw[line width=0.25pt,rounded corners=1pt]
    %     (15:3pt) -- (45:4pt) -- (89:5pt)--(135:6pt) -- (150:3pt) -- (182:4pt)--(235:6pt) -- (280:3pt) -- (330:2pt) -- cycle;
    \shade[black!50, draw, ultra thin, rounded corners=.5mm, top color=brown!20, bottom color=brown!20!gray]
                (15:3pt) -- (45:4pt) -- (89:5pt)--(135:6pt) -- (150:3pt) -- (182:4pt)--(235:6pt) -- (280:3pt) -- (330:2pt) -- cycle;
    \end{scope}
}

\NewDocumentCommand{\wave}{O{}}{%
    % Waves
    \setkeys{xarg}{x={0}, y={0}, scale={1}, rotate={0},#1}%

    %\tikzstyle{every node}=[font=\Huge]
    %\tikzstyle{every node/.style}=[scale=5]

    % Stone and wave
    \stone[x=20, y=20, scale=10]
    \draw (20, 16) arc (-90:-180:4);
    \draw (20, 15) arc (-90:-180:5);
    \draw (20, 14) arc (-90:-180:6);
    \draw (20, 10) arc (-90:-180:10);
    %\draw (20, 5) arc (-90:-180:15);
    % Not centered
    \draw (14, 8) arc (-116.565:-150:30);
    \draw (14, 8) arc (-116.565:-105:30);
    \draw (12, 4) arc (-116.565:-145:50);
    \draw (12, 4) arc (-116.565:-107:50);
}


\NewDocumentCommand{\floattwo}{O{}}{%
    \tincoord

    % Horizontal
    \draw[gb-neutral-blue] (-12,0) -- (16, 0);

    % Float
    \fishingfloat[x=10, y=0, scale=1, rotate=0]
    \fishingfloat[x=-10, y=0, scale=1, rotate=0]
}

\NewDocumentCommand{\floattriangle}{O{}}{%
    \tincoord

    % Last line
    \draw (15, -2.5) -- (-10, 10);

    \floattwo

    % Angle
    \draw pic [draw, ->, "$\theta$", angle eccentricity=1.4] {angle = X--Y--cross};
    \draw pic [draw, ->, "$\theta$", angle eccentricity=1.4] {angle = X2--X--S};
}


\NewDocumentCommand{\drawfishingfloatd}{}{%
    % To embed in scope
    % Triangle
    \draw (0, -3, 0) -- (1, 1, 0);
    \draw (0, -3, 0) -- (-1, 1, 0);
    % 2 lines
    \draw (-1, 1, 0) -- (1, 1, 0);
    \draw (-0.9375, 0.75, 0) -- (0.9375, 0.75, 0);
    % Top bottom
    \draw (0, 2, 0) -- (0, 2.5, 0);
    \draw (0, -3, 0) -- (0, -3.5, 0);
    % Arc: start angle, end angle, radius
    \draw (1, 1, 0) arc (0:180:1);
}

\NewDocumentCommand{\fishingfloatd}{O{}}{%
    % Fishing float
    % Size: 2, 6
    \setkeys{xarg}{x={0}, y={0}, scale={1}, rotate={0},
        xshift={}, yshift={}, #1}%
    \begin{scope} :xshift={\xxshift}
        [shift={(\xx, \xy)}, scale=\xscale, rotate=\xrotate]
        \drawfishingfloatd
    \end{scope}
}

\newcommand\drawwaved[3]{
    % Argument xorigin
    \draw [gb-neutral-blue, variable=\x, samples at={0, 0.01, ..., #3}]
    plot (
        { #1  },
        { 3 * cos( ( #2 +  \x/0.4*2) * pi r) },
        { 10 * \x  * sin(60) },
    ) ;
}

\newcommand\drawwavedtwo[3]{
    % Argument xorigin
    \draw [gb-neutral-blue, variable=\x, samples at={0, 0.01, ..., #3}]
    plot (
        { #1  },
        { 3 * cos( ( #2 +  \x/0.4*2) * pi r) },
        { 10 * \x  * sin(60) },
    ) ;
}
